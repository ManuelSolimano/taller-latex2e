% !TEX engine = xelatex

\documentclass[xetex]{beamer}
\usepackage{xltxtra}
\usepackage[twitter]{xelatexemoji}
\usepackage{amsmath}
\usepackage{amsfonts}
\usepackage{url}
\usepackage{listings}
\usepackage[spanish, mexico]{babel}
% \mode<presentation>
% {
%   \usetheme{Warsaw}
%   % or ...
%
%   \setbeamercovered{transparent}
%   % or whatever (possibly just delete it)
% }

\title{Mini-taller de tipografía digital}

\subtitle{Introducción a \LaTeXe}

\author{Manuel Solimano\inst{1} \\ \url{masolimano@uc.cl}}
% - Give the names in the same order as the appear in the paper.
% - Use the \inst{?} command only if the authors have different
%   affiliation.

\institute
{
  \inst{1}%
  Facultad de Física\\
  Pontificia Universidad Católica de Chile
}
% - Use the \inst command only if there are several affiliations.
% - Keep it simple, no one is interested in your street address.

\date{\today}
% - Either use conference name or its abbreviation.
% - Not really informative to the audience, more for people (including
%   yourself) who are reading the slides online



% If you have a file called "university-logo-filename.xxx", where xxx
% is a graphic format that can be processed by latex or pdflatex,
% resp., then you can add a logo as follows:

% \pgfdeclareimage[height=0.5cm]{university-logo}{logo.pdf}
% \logo{\pgfuseimage{university-logo}}

\begin{document}

\begin{frame}
  \titlepage
\end{frame}

\begin{frame}{Contenidos}
  \tableofcontents
  % You might wish to add the option [pausesections]
\end{frame}

\section{¿Qué es \TeX?}
\begin{frame}{Preámbulo histórico}
\begin{itemize}
    \item \TeX~es un sofisticado sistema de tipografía digital basado en lenguaje de marcas 💻
    \item Desarrollado por Donald Knuth en 1978 👴
    \item Produce documentos de alta calidad con excelente soporte para fórmulas matemáticas 📝
\end{itemize}
\end{frame}

\begin{frame}
  \begin{figure}
    \centering
    \lstset{language=TeX,frame=single}
   \scriptsize
    \lstinputlisting{code/math.tex}
 \caption{Ejemplo de código \TeX.  \emph{Fuente: The Lone \TeX{}nician.}}
    \label{fig:texcode}
  \end{figure}
\end{frame}

\begin{frame}{Pero era un infierno... 😵}
    Entonces Leslie Lamport crea \LaTeX
    \begin{itemize}
        \item Mucho más fácil de usar
    \end{itemize}
\end{frame}

\end{document}
